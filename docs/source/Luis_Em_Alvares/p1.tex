\documentclass{beamer}
\usepackage[spanish]{babel}
\usepackage[utf8]{inputenc}
\usetheme{Warsaw}
\usecolortheme{crane}
\useoutertheme{shadow}
\useinnertheme{rectangles}

\title[santosg572@gmail.com]{RAZONES Y PROPORCIONES}
%\subtitle{Dando nombres a los animales}
\author[L. González-Santos]{
L. González-Santos$^{1}$}
\institute[EDEN \& HELL]{
  $^{1}$
  Instituto de Neurobiología, UNAM\\
  Campus Juriquilla, Qro.
  \and
  \texttt{$^{1}$lgs@unam.mx}
}
\date{\today}

\begin{document}

\frame{\titlepage}

\begin{frame}
    \frametitle{RAZÖN}

En matemáticas una razón es la comparación de dos cantidades, por medio de división o cociente. 

\hfill

La razón entre $a$ y $b$, cuando $b$ es un número distinto de 
cero, se escribe:

$$
\frac{a}{b} \text{   o } a:b \text{ y se lee 'a es a b '}
$$

Por ejemplo, la razón entre 6 y 5 se escribe:

$$
\frac{6}{5} \text{ o } 6:5 \text{y se le 'seis es a cinco' }
$$

'a' - El numerador recibe el nombre de \textbf{antecedente}

\hfill

'b' - El denominador recibe el nombre de \textbf{consecuente}

\end{frame}


\begin{frame}
    \frametitle{¿CÓMO CALCULAMOS UNA RAZÓN?}

Calcular una razón, significa determinar el valor de ésta, el que se establece haciendo la división

entre el antecedente y el consecuente.

Ejemplos:


\end{frame}


\begin{frame}
    \frametitle{RAZÖN}


\end{frame}

\begin{frame}
    \frametitle{RAZÖN}


\end{frame}

\begin{frame}
    \frametitle{RAZÖN}


\end{frame}

\begin{frame}
    \frametitle{RAZÖN}


\end{frame}

\begin{frame}
    \frametitle{RAZÖN}


\end{frame}

\begin{frame}
    \frametitle{RAZÖN}


\end{frame}

\begin{frame}
    \frametitle{RAZÖN}


\end{frame}

\begin{frame}
    \frametitle{RAZÖN}


\end{frame}





\end{document}


