\documentclass{beamer}
\usepackage[spanish]{babel}
\usepackage[utf8]{inputenc}
\usetheme{Warsaw}
\usecolortheme{crane}
\useoutertheme{shadow}
\useinnertheme{rectangles}

\title[santosg572@gmail.com]{RAZONES Y PROPORCIONES}
%\subtitle{Dando nombres a los animales}
\author[L. González-Santos]{
L. González-Santos$^{1}$}
\institute[EDEN \& HELL]{
  $^{1}$
  Instituto de Neurobiología, UNAM\\
  Campus Juriquilla, Qro.
  \and
  \texttt{$^{1}$lgs@unam.mx}
}
\date{\today}

\begin{document}

\frame{\titlepage}

\begin{frame}
    \frametitle{RAZÖN}

En matemáticas una razón es la comparación de dos cantidades, por medio de división o cociente. 

\hfill

La razón entre $a$ y $b$, cuando $b$ es un número distinto de 
cero, se escribe:

$$
\frac{a}{b} \text{   o } a:b \text{ y se lee 'a es a b '}
$$

Por ejemplo, la razón entre 6 y 5 se escribe:

$$
\frac{6}{5} \text{ o } 6:5 \text{y se le 'seis es a cinco' }
$$

'a' - El numerador recibe el nombre de \textbf{antecedente}

\hfill

'b' - El denominador recibe el nombre de \textbf{consecuente}

\end{frame}


\begin{frame}
    \frametitle{¿CÓMO CALCULAMOS UNA RAZÓN?}

Calcular una razón, significa determinar el valor de ésta, el que se establece haciendo la división

entre el antecedente y el consecuente.

\hfill

Ejemplos:

\hfill

\begin{enumerate}
\item $\frac{1}{2} \to $ 1:2= 0.5 
\item $\frac{100}{50} \to $ 100:50 = 2
\end{enumerate}

\end{frame}


\begin{frame}
    \frametitle{RAZÖN}

Resuelva de acuerdo con lo solicitado en cada caso.

\begin{enumerate}
\item 7 y 5
\item 20 y 80
\item Antecedente 200 y consecuente 300
\item Antecedente 5 y consecuente 3
\end{enumerate}

\end{frame}

\begin{frame}
    \frametitle{Proporciones}

¿QUÉ ES UNA PROPORCIÓN?

\hfill

Una proporción es la igualdad entre dos o más
razones. Se escribe:

$$
\frac{a}{b} = \frac{c}{d}=k o a:b = c:d = k, con b,d \neq 0
$$

Se lee: « a es a b como c es a d »

k: Constante de proporcionalidad

a, d : Se denominan extremos de la proporción.

b, c : Se denominan medios de la proporción.

\hfill

Ejemplos:

\begin{enumerate}
\item $\frac{7}{3} = \frac{14}{6}= 2.3$
\item $\frac{10}{50} = \frac{5}{25} = \frac{15}{75} = \frac{1}{5} = 0.2 
\end{enumerate}

TEOREMA FUNDAMENTAL DE LAS PROPORCIONES (TFP)

El Teorema Fundamental de las Proporciones
dice que: En una proporción, el producto de los
extremos es igual al producto de los medios:

$$
\frac{a}{b} = \frac{c}{d} \to a*d = b*c, con b,d \neq 0
$$

Recíprocamente: Dos productos iguales pueden
escribirse como una proporción:

$$
a*d = b*c \to \frac{a}{b} = \frac{c}{d}, con b,d \neq 0
$$


\hfill

Ejemplos

\begin{enumerate}
\item $\frac{3}{4}= \frac{9}{12} \to 3*12 = 4*9
\end{enumerate}


\end{frame}

\begin{frame}
    \frametitle{RAZÖN}


\end{frame}

\begin{frame}
    \frametitle{RAZÖN}


\end{frame}

\begin{frame}
    \frametitle{RAZÖN}


\end{frame}

\begin{frame}
    \frametitle{RAZÖN}


\end{frame}

\begin{frame}
    \frametitle{RAZÖN}


\end{frame}

\begin{frame}
    \frametitle{RAZÖN}


\end{frame}





\end{document}


