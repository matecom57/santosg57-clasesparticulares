\begin{frame}
    \frametitle{EJERCICIOS RESUELTOS DE RAZONES}


\begin{enumerate}
\item En un curso, la razón entre la cantidad de hombres y de mujeres es 3 : 2. Si hay 24 hombres, ¿cuántos estudiantes hay en 
total en el curso?

\hfill

{\small
Respuesta:
}

\hfill

Hay 40 estudiantes en el curso, 24 hombres y 16 mujeres.


\item Un gásfiter y su ayudante, reciben por la instalación de tres sanitarios $ 270.000, los que se reparten en la razón 7 : 
2, 
¿cuánto dinero recibirá cada uno?

\hfill

Respuesta:

\hfill

Pago al ayudante: $ 60.000  Pago del gásfiter: $ 210.000

\item El perímetro de una cancha de fútbol mide 432 metros. Si la razón entre el ancho y el largo es 5 : 7, ¿cuánto mide cada 
lado de la cancha?

\hfill

Respuesta:

\hfill
El ancho de la cancha es de 90 metros y el largo 126 metros.

\item Las medidas de los ángulos interiores de un triángulo están en la razón 4 : 15 : 17 ¿Cuánto mide cada uno de los 
ángulos?

\hfill

Respuesta:

\hfill

La medida de los ángulos interiores es α= 20º, β = 75º, γ= 85º

\end{enumerate}

\end{frame}
